\documentclass{article}
\usepackage{multicol}
\usepackage{calc}
\usepackage{ifthen}
\usepackage[landscape]{geometry}
\usepackage{amsmath,amsthm,amsfonts,amssymb}
\usepackage{color,graphicx,overpic}
\usepackage{hyperref}
\usepackage{setspace}
\usepackage{tikz}
\usepackage{tcolorbox}

\linespread{1.3}

\hypersetup{
	pdftitle={Neel Patel's MATH 2004 (Multivariable Calculus) Equation Sheet}
}
\pdfinfo{
  /Title (MATH2004Equations.pdf)
  /Creator (TeX)
  /Producer (pdfTeX 3.14)
  /Author (Neel Patel)
  }

% This sets page margins to .5 inch if using letter paper, and to 1cm
% if using A4 paper. (This probably isn't strictly necessary.)
% If using another size paper, use default 1cm margins.
\ifthenelse{\lengthtest { \paperwidth = 11in}}    
	{ \geometry{top=.5in,left=.5in,right=.5in,bottom=.5in} }
    {\ifthenelse{ \lengthtest{ \paperwidth = 297mm}}
        {\geometry{top=1cm,left=1cm,right=1cm,bottom=1cm} }
        {\geometry{top=1cm,left=1cm,right=1cm,bottom=1cm} }
	}

	
% Turn off header and footer
\pagestyle{empty}

% Redefine section commands to use less space

\makeatletter
\renewcommand{\section}{\@startsection{section}{1}{0mm}%
                                {-1ex plus -.5ex minus -.2ex}%
                                {0.5ex plus .2ex}%x
                                {\normalfont\large\bfseries}}
\renewcommand{\subsection}{\@startsection{subsection}{2}{0mm}%
                                {-1explus -.5ex minus -.2ex}%
                                {0.5ex plus .2ex}%
                                {\normalfont\normalsize\bfseries}}
\renewcommand{\subsubsection}{\@startsection{subsubsection}{3}{0mm}%
                                {-1ex plus -.5ex minus -.2ex}%
                                {1ex plus .2ex}%
                                {\normalfont\small\bfseries}}
\makeatother

% Define BibTeX command
\def\BibTeX{{\rm B\kern-.05em{\sc i\kern-.025em b}\kern-.08em
    \kern-.1667em\lower.7ex\hbox{E}\kern-.125emX}}

% Don't print section numbers
\setcounter{secnumdepth}{0}

\setlength{\parindent}{0pt}
\setlength{\parskip}{0pt plus 0.5ex}

%My Environments
\newtheorem{example}[section]{Example}

% -----------------------------------------------------------------------

\begin{document}
\raggedright
\footnotesize
\begin{multicols}{3}


% multicol parameters
% These lengths are set only within the two main columns
		\setlength{\columnseprule}{0.25pt}
\setlength{\premulticols}{1pt}
\setlength{\postmulticols}{1pt}
\setlength{\multicolsep}{1pt}
\setlength{\columnsep}{2pt}

\begin{center}
     \normalsize{\underline{MATH 2004: Multivariable Calculus Equations}} \\
	 \normalsize{By Neel Patel} \\
\end{center}
\section{Chapter 1}
\subsection{Trig Identities}
$cosh\,t = \frac{1}{2}(e^t + e^{-t})$ \\
$sinh\,t = \frac{1}{2}(e^t - e^{-t})$ \\
$cos^2x = \frac{1+cos2x}{2}$ \\
$sin^2x = \frac{1-cos2x}{2}$ \\
$2sinxcosx = sin2x$ \\
$sinasinb = \frac{cos(a-b)-cos(a+b)}{2}$ \\
$cosacosb = \frac{cos(a+b)+cos(a-b)}{2}$ \\
$sinacosb = \frac{sin(a+b)+sin(a-b)}{2}$ \\
\textbf{Trig Integral} \\
$\int sin^m xcos^n xdx:$ \\
- If m is odd, let $u = cosx$ \\
- If n is odd, let $u = sinx$ \\
- If both are even, then use half-angle formula \\
\textbf{Half-angle Formula} \\
$sin(\frac{\alpha}{2}) = \pm\sqrt{\frac{1-cos\alpha}{2}}$ \\
$cos(\frac{\alpha}{2}) = \pm\sqrt{\frac{1+cos\alpha}{2}}$ \\
\textbf{Trig Substitution} \\
$\sqrt{a^2-x^2} \to x = asin\theta, \-\frac{\pi}{2}\leq\theta\leq\frac{\pi}{2}$ \\
$\sqrt{a^2+x^2} \to x = atan\theta, \-\frac{\pi}{2}\leq\theta\leq\frac{\pi}{2}$ \\
$\sqrt{x^2-a^2} \to x = asec\theta, \-\frac{\pi}{2}\leq\theta\leq\frac{\pi}{2}$ \\
\subsection{1.1-1.6}
\textbf{Length (norm,magnitude)} $|(a,b,c)| = \sqrt{a^2+b^2+c^2}$ \\
\textbf{Unit Vector} $|\vec{u}| = 1$ \\
\subsection{1.7-1.10}
\textbf{Dot Product} $\vec{u} \cdot \vec{v} = u_{1}v_{1} + u_{2}v_{2} + u_{3}v_{3}$ \\
\textbf{Orthogonal} $\vec{u}\perp\vec{v}$ if $\vec{u} \cdot \vec{v} = 0$ \\
\textbf{Angle} $cos\theta = \frac{\vec{u} \cdot \vec{v}}{|\vec{u}||\vec{v}|}, 0\leq\theta\pi$ \\
$cos\alpha = \frac{\vec{u} \cdot \vec{i}}{|\vec{u}||\vec{i}|}$ \\
$cos\beta = \frac{\vec{u} \cdot \vec{j}}{|\vec{u}||\vec{j}|}$ \\
$cos\gamma = \frac{\vec{u} \cdot \vec{k}}{|\vec{u}||\vec{k}|}$ \\
$cos^2\alpha + cos^2\beta + cos^2\gamma = 1, \frac{\vec{u}}{|\vec{u}|} = (cos\alpha,cos\beta,cos\gamma)$ \\
\textbf{Cross Product} $\vec{u} \times \vec{v} = (u_{2}v_{3} - u_{3}v_{2}, u_{3}v_{1} - u_{1}v_{3}, u_{1}v_{3} - u_{2}v_{1})$\\
\textbf{Area of Parallelogram} $A = |\vec{u} \times \vec{v}|$ \\
\textbf{Area of Triangle} $A = \frac{1}{2}|\vec{u} \times \vec{v}|$ \\
\textbf{Volume of Parallelopiped} $V = |\vec{u} \cdot (\vec{v} \times \vec{w})|$ \\

\section{Chapter 2}
\subsection{2.1-2.5 Lines and Planes}
\textbf{Equation of Line} 
$\vec{r}(t) = P + t\vec{v}$ \\
\textbf{Line Segment} 
$\vec{r}(t) = (1-t)\vec{P} + t\vec{Q}$\\

\subsection{2.6 Rotations/Translations in Plane}
\textbf{Counter-Clockwise} \\
$\begin{pmatrix}
	x' \\
	y'
\end{pmatrix} 
= A\begin{pmatrix}
	x \\
	y
\end{pmatrix}
= \begin{pmatrix}
	xcos\theta - ysin\theta \\
	xsin\theta + ycos\theta
\end{pmatrix},
A = \begin{pmatrix}
		cos\theta & -sin\theta \\
		sin\theta & cos\theta
\end{pmatrix}$
$A^{-1} = \begin{pmatrix}
		cos\theta & sin\theta \\
		-sin\theta & cos\theta
\end{pmatrix}$ \\
\textbf{Translated Origin} \\
$\begin{pmatrix}
	x' \\
	y'
\end{pmatrix} 
= \begin{pmatrix}
		x-h \\
		y-k
\end{pmatrix}$

\subsection{2.7-2.8 Parametric Curves}
\textbf{Parabola} $(x-x_{0})^2 = 4p(y-y_{0})$, $y = \frac{1}{4p}(t-x_{0})^2+y_{0}$ \\
\textbf{Ellipse} $\frac{(x-x_{0})^2}{a^2}+\frac{(y-y_{0})^2}{a^2} = 1$, $x = x_{0} + a\,cos\,t, y=y_{0}+b\,sin\,t$ \\
\textbf{Hyperbola} $\frac{(x-x_{0})^2}{a^2}-\frac{(y-y_{0})^2}{a^2} = 1$, $x = x_{0} + a\,sec\,t, y = y_{0} + b\,tan\,t, \pi<t<\pi$\\
\textbf{General Conic Sections} \\
If $B^2-4AC=0$, either parabola, 2 parallel lines, 1 line, or no curve \\
If $B^2-4AC>0$, either hyperbola, or 2 intersecting lines \\
If $B^2-4AC<0$, either ellipse, circle, point, or no curve \\
\textbf{Derivatives} \\
$\frac{dy}{dx} = \frac{\frac{dy}{dt}}{\frac{dx}{dt}}$ \\
$\frac{d^2y}{dx^2} = \frac{\frac{d}{dt}(\frac{dy}{dx})}{\frac{dx}{dt}}$

\subsection{2.9 Applications to Area Problems}
\textbf{Area under Parametric Curve}
\[ A = \int_{\alpha}^{\beta} y(t)x'(t) \,dt \]
\textbf{Area of region R enclosed by C}
\[ A =| \int_{\alpha}^{\beta} y(t)x'(t) \,dt|\]
\textbf{Arc Length}
\[ L = \int_{\alpha}^{\beta} \sqrt{[x'(t)]^2+[y'(t)]^2}\,dt \] 

\subsection{2.11-2.14 Polar Coordinates}
\textbf{Polar $\rightleftharpoons$ Cartesian} \\
$x = rcos\theta, y = rsin\theta, r = \sqrt{x^2+y^2}, tan\theta = \frac{y}{x}$ \\
\textbf{Derivative} \\
$\frac{dy}{dx} = \frac{r'_{\theta}sin\theta+rcos\theta}{r'_{\theta}cos\theta-rsin\theta}$ \\
\textbf{Area of region bounded by $r = f(\theta)$}
\[ A = \int_{\alpha}^{\beta} \frac{1}{2}r^2\,d\theta \]
\textbf{Area of more general region}
\[ A = \int_{\alpha}^{\beta} \frac{1}{2}(r_{o}^2-r_{i}^2)\,d\theta \]
\textbf{Arc Length}
\[ L = \int_{\alpha}^{\beta} \sqrt{(r'_{\theta})^2 + r^2}\,d\theta \]

\section{Chapter 3}
\subsection{3.2-3.3 Partial Derivatives}
\[ z_{x}=\frac{\partial z}{\partial x} := \frac{\partial f}{\partial x} := D_{x}f := \lim_{h \to 0}\frac{f(x+h,y)-f(x,y)}{h} \]
\[ z_{y}=\frac{\partial z}{\partial y} := \frac{\partial f}{\partial y} := D_{y}f := \lim_{h \to 0}\frac{f(x, y + h)-f(x,y)}{h} \]
\[ f_{x}(x,y,z) \]
\[ = \frac{\partial w}{\partial x} := \frac{\partial f}{\partial x} := D_{x}f := \lim_{h \to 0}\frac{f(x+h,y,z)-f(x,y,z)}{h} \]
\subsection{3.5 Directional Derivatives and Gradients}
$ f_{\vec{u}}(x_{0},y_{0})\;or\;D_{\vec{u}}f(x_{0},y_{0}) = (f_{x}(x_{0},y_{0})u_{1} + f_{y}(x_{0},y_{0})u_{2}) $ \\
$ \nabla f(x_{0},y_{0}) = (f_{x}(x_{0},y_{0})) $ \\
$ f_{\vec{u}}(x_{0},y_{0},z_{0})= (f_{x}(x_{0},y_{0},z_{0})u_{1} + f_{y}(x_{0},y_{0},z_{0})u_{2} + f_{z}(x_{0},y_{0},z_{0})u_{3})) $\\
$ \nabla f(x_{0},y_{0},z_{0}) = (f_{x}(x_{0},y_{0},z_{0})) $ \\


\end{multicols}
\end{document}
